

%OLD 
%As we move toward pre-Exascale systems, two of the DOE leadership class systems will consist of very powerful OpenPOWER compute nodes which will be more complex to program. These systems will have massive amounts of parallelism; where threads may be running on POWER9 cores as well as on accelerators. Advances in memory interconnects, such as NVLINK, will provide a unified shared memory address spaces for different types of memories HBM, DRAM, etc. In preparation for such system, we need to improve our understanding on how OpenMP supports the concept of affinity as well as memory placement on POWER8 systems. Data locality and affinity are key program optimizations to exploit the compute and memory capabilities to achieve good performance by minimizing data motion across NUMA domains and access the cache efficiently. This paper is the first step to evaluate the current features of OpenMP 4.0 on the POWER8 processors, and on how to measure its effects on a system with two POWER8 sockets. We experiment with the different affinity settings provided by OpenMP 4.0 to quantify the costs of having good data locality vs not,  and measure their effects via hardware counters. We also find out which affinity settings benefits more from data locality. Based on this study we describe the current state of art, the challenges we faced in quantifying effects of affinity, and and ideas on how OpenMP 5.0 should be improved to address affinity in the context of NUMA domains and accelerators.


\let\thefootnote\relax\footnote{This manuscript has been authored by UT-Battelle, LLC under Contract No. DE-AC05-00OR22725 with the U.S. Department of Energy. The United States Government retains and the publisher, by accepting the article for publication, acknowledges that the United States Government retains a non-exclusive, paid-up, irrevocable, worldwide license to publish or reproduce the published form of this manuscript, or allow others to do so, for United States Government purposes. The Department of Energy will provide public access to these results of federally sponsored research in accordance with the DOE Public Access Plan (http://energy.gov/downloads/doe-public-access-plan).}
